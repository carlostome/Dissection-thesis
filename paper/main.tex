
\documentclass[sigplan,10pt,review]{acmart}

%% For double-blind review submission, w/ CCS and ACM Reference
%\documentclass[sigplan,10pt,review,anonymous]{acmart}\settopmatter{printfolios=true}
%% For single-blind review submission, w/o CCS and ACM Reference (max submission space)
%\documentclass[sigplan,10pt,review]{acmart}\settopmatter{printfolios=true,printccs=false,printacmref=false}
%% For single-blind review submission, w/ CCS and ACM Reference
%\documentclass[sigplan,10pt,review]{acmart}\settopmatter{printfolios=true}
%% For final camera-ready submission, w/ required CCS and ACM Reference
%\documentclass[sigplan,10pt]{acmart}\settopmatter{}

\settopmatter{printfolios=true,printccs=false,printacmref=false}

% TODO notes
%\usepackage[draft]{todonotes}
\usepackage{xcolor}
\newcommand\todo[1]{\textcolor{red}{\textbf{TODO:~}#1}}


%% Agda stuff
\usepackage[conor]{agda}
\newcommand{\AK}{\AgdaKeyword}
\newcommand{\AY}{\AgdaSymbol}
\newcommand{\AN}{\AgdaNumber}
\newcommand{\AS}{\AgdaSpace}
\newcommand{\AB}{\AgdaBound}
\newcommand{\AO}{\AgdaOperator}
\newcommand{\AI}{\AgdaInductiveConstructor}
\newcommand{\AC}{\AgdaCoinductiveConstructor}
\newcommand{\AD}{\AgdaDatatype}
\newcommand{\AF}{\AgdaFunction}
\newcommand{\AM}{\AgdaModule}
\newcommand{\AL}{\AgdaField}
\newcommand{\AR}{\AgdaArgument}
\newcommand{\AT}{\AgdaIndent}
\newcommand{\ARR}{\AgdaRecord}
\newcommand{\AP}{\AgdaPostulate}
\newcommand{\APT}{\AgdaPrimitiveType}


\newcommand{\nonterm}[1]{\hspace*{-0.1cm}\colorbox{orange!25}{#1}}
\newcommand{\hole}[1]{\colorbox{yellow!50}{\ensuremath{\bigbox_{#1}}}}
%% Conference information
%% Supplied to authors by publisher for camera-ready submission;
%% use defaults for review submission.
\acmConference[TyDe'18]{ACM SIGPLAN Workshop on Type-driven
Development (TyDe)}{January 01--03, 2017}{St. Louis, MO, USA}
\acmYear{2018} \acmISBN{} % \acmISBN{978-x-xxxx-xxxx-x/YY/MM}
\acmDOI{} % \acmDOI{10.1145/nnnnnnn.nnnnnnn} \startPage{1}

%% Copyright information
%% Supplied to authors (based on authors' rights management selection;
%% see authors.acm.org) by publisher for camera-ready submission;
%% use 'none' for review submission.
\setcopyright{none}
%\setcopyright{acmcopyright}
%\setcopyright{acmlicensed}
%\setcopyright{rightsretained}
%\copyrightyear{2017}           %% If different from \acmYear

%% Bibliography style
\bibliographystyle{ACM-Reference-Format}
%% Citation style
%\citestyle{acmauthoryear}  %% For author/year citations
%\citestyle{acmnumeric}     %% For numeric citations
%\setcitestyle{nosort}      %% With 'acmnumeric', to disable automatic
                            %% sorting of references within a single citation;
                            %% e.g., \cite{Smith99,Carpenter05,Baker12}
                            %% rendered as [14,5,2] rather than [2,5,14].
%\setcitesyle{nocompress}   %% With 'acmnumeric', to disable automatic
                            %% compression of sequential references within a
                            %% single citation;
                            %% e.g., \cite{Baker12,Baker14,Baker16}
                            %% rendered as [2,3,4] rather than [2-4].


%%%%%%%%%%%%%%%%%%%%%%%%%%%%%%%%%%%%%%%%%%%%%%%%%%%%%%%%%%%%%%%%%%%%%%
%% Note: Authors migrating a paper from traditional SIGPLAN
%% proceedings format to PACMPL format must update the
%% '\documentclass' and topmatter commands above; see
%% 'acmart-pacmpl-template.tex'.
%%%%%%%%%%%%%%%%%%%%%%%%%%%%%%%%%%%%%%%%%%%%%%%%%%%%%%%%%%%%%%%%%%%%%%
%% Some recommended packages.
\usepackage{booktabs}   %% For formal tables:
                        %% http://ctan.org/pkg/booktabs
\usepackage{subcaption} %% For complex figures with subfigures/subcaptions
                        %% http://ctan.org/pkg/subcaption
%% ODER: format ==         = "\mathrel{==}"
%% ODER: format /=         = "\neq "
%
%
\makeatletter
\@ifundefined{lhs2tex.lhs2tex.sty.read}%
  {\@namedef{lhs2tex.lhs2tex.sty.read}{}%
   \newcommand\SkipToFmtEnd{}%
   \newcommand\EndFmtInput{}%
   \long\def\SkipToFmtEnd#1\EndFmtInput{}%
  }\SkipToFmtEnd

\newcommand\ReadOnlyOnce[1]{\@ifundefined{#1}{\@namedef{#1}{}}\SkipToFmtEnd}
\usepackage{amstext}
\usepackage{amssymb}
\usepackage{stmaryrd}
\DeclareFontFamily{OT1}{cmtex}{}
\DeclareFontShape{OT1}{cmtex}{m}{n}
  {<5><6><7><8>cmtex8
   <9>cmtex9
   <10><10.95><12><14.4><17.28><20.74><24.88>cmtex10}{}
\DeclareFontShape{OT1}{cmtex}{m}{it}
  {<-> ssub * cmtt/m/it}{}
\newcommand{\texfamily}{\fontfamily{cmtex}\selectfont}
\DeclareFontShape{OT1}{cmtt}{bx}{n}
  {<5><6><7><8>cmtt8
   <9>cmbtt9
   <10><10.95><12><14.4><17.28><20.74><24.88>cmbtt10}{}
\DeclareFontShape{OT1}{cmtex}{bx}{n}
  {<-> ssub * cmtt/bx/n}{}
\newcommand{\tex}[1]{\text{\texfamily#1}}	% NEU

\newcommand{\Sp}{\hskip.33334em\relax}


\newcommand{\Conid}[1]{\mathit{#1}}
\newcommand{\Varid}[1]{\mathit{#1}}
\newcommand{\anonymous}{\kern0.06em \vbox{\hrule\@width.5em}}
\newcommand{\plus}{\mathbin{+\!\!\!+}}
\newcommand{\bind}{\mathbin{>\!\!\!>\mkern-6.7mu=}}
\newcommand{\rbind}{\mathbin{=\mkern-6.7mu<\!\!\!<}}% suggested by Neil Mitchell
\newcommand{\sequ}{\mathbin{>\!\!\!>}}
\renewcommand{\leq}{\leqslant}
\renewcommand{\geq}{\geqslant}
\usepackage{polytable}

%mathindent has to be defined
\@ifundefined{mathindent}%
  {\newdimen\mathindent\mathindent\leftmargini}%
  {}%

\def\resethooks{%
  \global\let\SaveRestoreHook\empty
  \global\let\ColumnHook\empty}
\newcommand*{\savecolumns}[1][default]%
  {\g@addto@macro\SaveRestoreHook{\savecolumns[#1]}}
\newcommand*{\restorecolumns}[1][default]%
  {\g@addto@macro\SaveRestoreHook{\restorecolumns[#1]}}
\newcommand*{\aligncolumn}[2]%
  {\g@addto@macro\ColumnHook{\column{#1}{#2}}}

\resethooks

\newcommand{\onelinecommentchars}{\quad-{}- }
\newcommand{\commentbeginchars}{\enskip\{-}
\newcommand{\commentendchars}{-\}\enskip}

\newcommand{\visiblecomments}{%
  \let\onelinecomment=\onelinecommentchars
  \let\commentbegin=\commentbeginchars
  \let\commentend=\commentendchars}

\newcommand{\invisiblecomments}{%
  \let\onelinecomment=\empty
  \let\commentbegin=\empty
  \let\commentend=\empty}

\visiblecomments

\newlength{\blanklineskip}
\setlength{\blanklineskip}{0.66084ex}

\newcommand{\hsindent}[1]{\quad}% default is fixed indentation
\let\hspre\empty
\let\hspost\empty
\newcommand{\NB}{\textbf{NB}}
\newcommand{\Todo}[1]{$\langle$\textbf{To do:}~#1$\rangle$}

\EndFmtInput
\makeatother
%
%
%
%
%
%
% This package provides two environments suitable to take the place
% of hscode, called "plainhscode" and "arrayhscode". 
%
% The plain environment surrounds each code block by vertical space,
% and it uses \abovedisplayskip and \belowdisplayskip to get spacing
% similar to formulas. Note that if these dimensions are changed,
% the spacing around displayed math formulas changes as well.
% All code is indented using \leftskip.
%
% Changed 19.08.2004 to reflect changes in colorcode. Should work with
% CodeGroup.sty.
%
\ReadOnlyOnce{polycode.fmt}%
\makeatletter

\newcommand{\hsnewpar}[1]%
  {{\parskip=0pt\parindent=0pt\par\vskip #1\noindent}}

% can be used, for instance, to redefine the code size, by setting the
% command to \small or something alike
\newcommand{\hscodestyle}{}

% The command \sethscode can be used to switch the code formatting
% behaviour by mapping the hscode environment in the subst directive
% to a new LaTeX environment.

\newcommand{\sethscode}[1]%
  {\expandafter\let\expandafter\hscode\csname #1\endcsname
   \expandafter\let\expandafter\endhscode\csname end#1\endcsname}

% "compatibility" mode restores the non-polycode.fmt layout.

\newenvironment{compathscode}%
  {\par\noindent
   \advance\leftskip\mathindent
   \hscodestyle
   \let\\=\@normalcr
   \let\hspre\(\let\hspost\)%
   \pboxed}%
  {\endpboxed\)%
   \par\noindent
   \ignorespacesafterend}

\newcommand{\compaths}{\sethscode{compathscode}}

% "plain" mode is the proposed default.
% It should now work with \centering.
% This required some changes. The old version
% is still available for reference as oldplainhscode.

\newenvironment{plainhscode}%
  {\hsnewpar\abovedisplayskip
   \advance\leftskip\mathindent
   \hscodestyle
   \let\hspre\(\let\hspost\)%
   \pboxed}%
  {\endpboxed%
   \hsnewpar\belowdisplayskip
   \ignorespacesafterend}

\newenvironment{oldplainhscode}%
  {\hsnewpar\abovedisplayskip
   \advance\leftskip\mathindent
   \hscodestyle
   \let\\=\@normalcr
   \(\pboxed}%
  {\endpboxed\)%
   \hsnewpar\belowdisplayskip
   \ignorespacesafterend}

% Here, we make plainhscode the default environment.

\newcommand{\plainhs}{\sethscode{plainhscode}}
\newcommand{\oldplainhs}{\sethscode{oldplainhscode}}
\plainhs

% The arrayhscode is like plain, but makes use of polytable's
% parray environment which disallows page breaks in code blocks.

\newenvironment{arrayhscode}%
  {\hsnewpar\abovedisplayskip
   \advance\leftskip\mathindent
   \hscodestyle
   \let\\=\@normalcr
   \(\parray}%
  {\endparray\)%
   \hsnewpar\belowdisplayskip
   \ignorespacesafterend}

\newcommand{\arrayhs}{\sethscode{arrayhscode}}

% The mathhscode environment also makes use of polytable's parray 
% environment. It is supposed to be used only inside math mode 
% (I used it to typeset the type rules in my thesis).

\newenvironment{mathhscode}%
  {\parray}{\endparray}

\newcommand{\mathhs}{\sethscode{mathhscode}}

% texths is similar to mathhs, but works in text mode.

\newenvironment{texthscode}%
  {\(\parray}{\endparray\)}

\newcommand{\texths}{\sethscode{texthscode}}

% The framed environment places code in a framed box.

\def\codeframewidth{\arrayrulewidth}
\RequirePackage{calc}

\newenvironment{framedhscode}%
  {\parskip=\abovedisplayskip\par\noindent
   \hscodestyle
   \arrayrulewidth=\codeframewidth
   \tabular{@{}|p{\linewidth-2\arraycolsep-2\arrayrulewidth-2pt}|@{}}%
   \hline\framedhslinecorrect\\{-1.5ex}%
   \let\endoflinesave=\\
   \let\\=\@normalcr
   \(\pboxed}%
  {\endpboxed\)%
   \framedhslinecorrect\endoflinesave{.5ex}\hline
   \endtabular
   \parskip=\belowdisplayskip\par\noindent
   \ignorespacesafterend}

\newcommand{\framedhslinecorrect}[2]%
  {#1[#2]}

\newcommand{\framedhs}{\sethscode{framedhscode}}

% The inlinehscode environment is an experimental environment
% that can be used to typeset displayed code inline.

\newenvironment{inlinehscode}%
  {\(\def\column##1##2{}%
   \let\>\undefined\let\<\undefined\let\\\undefined
   \newcommand\>[1][]{}\newcommand\<[1][]{}\newcommand\\[1][]{}%
   \def\fromto##1##2##3{##3}%
   \def\nextline{}}{\) }%

\newcommand{\inlinehs}{\sethscode{inlinehscode}}

% The joincode environment is a separate environment that
% can be used to surround and thereby connect multiple code
% blocks.

\newenvironment{joincode}%
  {\let\orighscode=\hscode
   \let\origendhscode=\endhscode
   \def\endhscode{\def\hscode{\endgroup\def\@currenvir{hscode}\\}\begingroup}
   %\let\SaveRestoreHook=\empty
   %\let\ColumnHook=\empty
   %\let\resethooks=\empty
   \orighscode\def\hscode{\endgroup\def\@currenvir{hscode}}}%
  {\origendhscode
   \global\let\hscode=\orighscode
   \global\let\endhscode=\origendhscode}%

\makeatother
\EndFmtInput
%

%% Agda keywords

%% Agda standard types

%% Constructors of the above types

%% Some standard functions
% format map    = "\AF{map}"       % Disabled

%% Non-colored stuff

%% Useful symbols


% fontsize of code snippets
\renewcommand\hscodestyle{%
   \setlength\leftskip{0.25cm}%
   \footnotesize
}

\begin{document}

\title{Dissection: verified and terminating}

\author{Carlos Tom\'e Corti\~nas}
\affiliation{
  \department{Department of Information and Computing Sciences}  
  \institution{University of Utrecht}
  \country{The Netherlands}          
}
\email{first1.last1@inst1.edu}      

\author{Wouter Swierstra}
\affiliation{
  \department{Department of Information and Computing Sciences}
  \institution{University of Utrecht}
  \country{The Netherlands}
}
\email{w.s.swierstra@uu.nl}


\begin{abstract}
  Lorem ipsum sic dolor amet.  Lorem ipsum sic dolor amet.  Lorem
  ipsum sic dolor amet.  Lorem ipsum sic dolor amet.  Lorem ipsum sic
  dolor amet.  Lorem ipsum sic dolor amet.  Lorem ipsum sic dolor
  amet.
  \todo{Write abstract}
\end{abstract}

%% 2012 ACM Computing Classification System (CSS) concepts
%% Generate at 'http://dl.acm.org/ccs/ccs.cfm'.
\begin{CCSXML}
<ccs2012>
<concept>
<concept_id>10011007.10011006.10011008.10011009.10011012</concept_id>
<concept_desc>Software and its engineering~Functional languages</concept_desc>
<concept_significance>500</concept_significance>
</concept>
<concept>
<concept_id>10011007.10011006.10011008.10011024.10011028</concept_id>
<concept_desc>Software and its engineering~Data types and structures</concept_desc>
<concept_significance>300</concept_significance>
</concept>
<concept>
<concept_id>10011007.10011006.10011008.10011024.10011033</concept_id>
<concept_desc>Software and its engineering~Recursion</concept_desc>
<concept_significance>300</concept_significance>
</concept>
<concept>
<concept_id>10003752.10003753.10010622</concept_id>
<concept_desc>Theory of computation~Abstract machines</concept_desc>
<concept_significance>300</concept_significance>
</concept>
<concept>
<concept_id>10003752.10003790.10011740</concept_id>
<concept_desc>Theory of computation~Type theory</concept_desc>
<concept_significance>300</concept_significance>
</concept>
<concept>
<concept_id>10003752.10003790.10002990</concept_id>
<concept_desc>Theory of computation~Logic and verification</concept_desc>
<concept_significance>100</concept_significance>
</concept>
</ccs2012>
\end{CCSXML}

\ccsdesc[500]{Software and its engineering~Functional languages}
\ccsdesc[300]{Software and its engineering~Data types and structures}
\ccsdesc[300]{Software and its engineering~Recursion}
\ccsdesc[300]{Theory of computation~Abstract machines}
\ccsdesc[300]{Theory of computation~Type theory}
\ccsdesc[100]{Theory of computation~Logic and verification}

%% End of generated code


\keywords{datatype generic programming, catamorphisms, dissection,
  dependent types, Agda, well-founded recursion \todo{Keywords?}}

\maketitle


% Bound variables

\section{Introduction}

Folds, or \emph{catamorphisms}, are a pervasive
programming pattern. A fold generalizes many simple traversals over an
algebraic data type. Functions implemented by means of a fold are
\emph{compositional} and structurally recursive. Consider, for
instance, the following expression data type, written in the
dependently typed programming language Agda\todo{citation}:

\begin{hscode}\SaveRestoreHook
\column{B}{@{}>{\hspre}l<{\hspost}@{}}%
\column{3}{@{}>{\hspre}l<{\hspost}@{}}%
\column{5}{@{}>{\hspre}l<{\hspost}@{}}%
\column{10}{@{}>{\hspre}l<{\hspost}@{}}%
\column{13}{@{}>{\hspre}l<{\hspost}@{}}%
\column{19}{@{}>{\hspre}l<{\hspost}@{}}%
\column{E}{@{}>{\hspre}l<{\hspost}@{}}%
\>[3]{}\AK{data}\;\AD{Expr}\;\mathbin{:}\;\AP{Set}\;\AK{where}{}\<[E]%
\\
\>[3]{}\hsindent{2}{}\<[5]%
\>[5]{}\AI{Val}\;{}\<[10]%
\>[10]{}\mathbin{:}\;{}\<[13]%
\>[13]{}\AD{\ensuremath{\mathbb{N}}}\;{}\<[19]%
\>[19]{}\to \;\AD{Expr}{}\<[E]%
\\
\>[3]{}\hsindent{2}{}\<[5]%
\>[5]{}\AI{Add}\;{}\<[10]%
\>[10]{}\mathbin{:}\;{}\<[13]%
\>[13]{}\AD{Expr}\;{}\<[19]%
\>[19]{}\to \;\AD{Expr}\;\to \;\AD{Expr}{}\<[E]%
\ColumnHook
\end{hscode}\resethooks
%
We can write a write a simple evaluator, mapping expressions to
natural numbers, as follows:
%
\begin{hscode}\SaveRestoreHook
\column{B}{@{}>{\hspre}l<{\hspost}@{}}%
\column{3}{@{}>{\hspre}l<{\hspost}@{}}%
\column{21}{@{}>{\hspre}l<{\hspost}@{}}%
\column{E}{@{}>{\hspre}l<{\hspost}@{}}%
\>[3]{}\AF{eval}\;\mathbin{:}\;\AD{Expr}\;\to \;\AD{\ensuremath{\mathbb{N}}}{}\<[E]%
\\
\>[3]{}\AF{eval}\;(\AI{Val}\;\AB{n})\;{}\<[21]%
\>[21]{}\mathrel{=}\;\AB{n}{}\<[E]%
\\
\>[3]{}\AF{eval}\;(\AI{Add}\;\AB{\ensuremath{e_1}}\;\AB{\ensuremath{e_2}})\;{}\<[21]%
\>[21]{}\mathrel{=}\;\AF{eval}\;\AB{\ensuremath{e_1}}\;\AF{+}\;\AF{eval}\;\AB{\ensuremath{e_2}}{}\<[E]%
\ColumnHook
\end{hscode}\resethooks
%
In the case for \ensuremath{\AI{Add}\;\AB{\ensuremath{e_1}}\;\AB{\ensuremath{e_2}}}, the \ensuremath{\AF{eval}} function makes two recursive
calls and sums their results. Such a function can be implemented by a
fold, passing the addition and identity functions as the argument
algebra.

Unfortunately, all in the garden is not rosy. The operator \ensuremath{\AF{\_+\_}}
needs both of its parameters to be fully evaluated before it can
reduce further. When evaluating large expressions, the size of the
stack used during execution grows linearly with the size of the input,
potentially leading to a stack overflow, causing the execution of the
program to halt unexpectedly.

To address this problem, we can rewrite the evaluator to be
\emph{tail recursive}. Modern compilers typically map tail recursive
functions to machine code that runs in constant memory. To write such
a tail recursive function, we need to introduce an explicit stack
storing both intemediate results and the subtrees that have not yet
been evaluated. We can define such a stack as follows:

\begin{hscode}\SaveRestoreHook
\column{B}{@{}>{\hspre}l<{\hspost}@{}}%
\column{3}{@{}>{\hspre}l<{\hspost}@{}}%
\column{5}{@{}>{\hspre}l<{\hspost}@{}}%
\column{12}{@{}>{\hspre}l<{\hspost}@{}}%
\column{20}{@{}>{\hspre}l<{\hspost}@{}}%
\column{E}{@{}>{\hspre}l<{\hspost}@{}}%
\>[3]{}\AK{data}\;\AD{Stack}\;\mathbin{:}\;\AP{Set}\;\AK{where}{}\<[E]%
\\
\>[3]{}\hsindent{2}{}\<[5]%
\>[5]{}\AI{Top}\;{}\<[12]%
\>[12]{}\mathbin{:}\;\AD{Stack}{}\<[E]%
\\
\>[3]{}\hsindent{2}{}\<[5]%
\>[5]{}\AI{Left}\;{}\<[12]%
\>[12]{}\mathbin{:}\;\AD{Expr}\;{}\<[20]%
\>[20]{}\to \;\AD{Stack}\;\to \;\AD{Stack}{}\<[E]%
\\
\>[3]{}\hsindent{2}{}\<[5]%
\>[5]{}\AI{Right}\;{}\<[12]%
\>[12]{}\mathbin{:}\;\AD{\ensuremath{\mathbb{N}}}\;{}\<[20]%
\>[20]{}\to \;\AD{Stack}\;\to \;\AD{Stack}{}\<[E]%
\ColumnHook
\end{hscode}\resethooks

We can define a tail-recursive evaluation function by means of a
pair of mutually recursive functions \ensuremath{\AF{load}} and \ensuremath{\AF{unload}}. The \ensuremath{\AF{load}}
function traverses the input expression, pushing subtrees on the
stack; the \ensuremath{\AF{unload}} function unloads the stack, while accumulating a
(partial) result.

\begin{hscode}\SaveRestoreHook
\column{B}{@{}>{\hspre}l<{\hspost}@{}}%
\column{3}{@{}>{\hspre}l<{\hspost}@{}}%
\column{5}{@{}>{\hspre}l<{\hspost}@{}}%
\column{16}{@{}>{\hspre}l<{\hspost}@{}}%
\column{23}{@{}>{\hspre}l<{\hspost}@{}}%
\column{32}{@{}>{\hspre}l<{\hspost}@{}}%
\column{E}{@{}>{\hspre}l<{\hspost}@{}}%
\>[3]{}\AK{mutual}{}\<[E]%
\\
\>[3]{}\hsindent{2}{}\<[5]%
\>[5]{}\nonterm{\AF{load}}\;\mathbin{:}\;\AD{Expr}\;\to \;\AD{Stack}\;\to \;\AD{\ensuremath{\mathbb{N}}}{}\<[E]%
\\
\>[3]{}\hsindent{2}{}\<[5]%
\>[5]{}\AF{load}\;(\AI{Val}\;\AB{n})\;{}\<[23]%
\>[23]{}\AB{stk}\;\mathrel{=}\;\nonterm{\AF{unload}}\;\AB{n}\;\AB{stk}{}\<[E]%
\\
\>[3]{}\hsindent{2}{}\<[5]%
\>[5]{}\AF{load}\;(\AI{Add}\;\AB{\ensuremath{e_1}}\;\AB{\ensuremath{e_2}})\;{}\<[23]%
\>[23]{}\AB{stk}\;\mathrel{=}\;\nonterm{\AF{load}}\;\AB{\ensuremath{e_1}}\;(\AI{Left}\;\AB{\ensuremath{e_2}}\;\AB{stk}){}\<[E]%
\\[\blanklineskip]%
\>[3]{}\hsindent{2}{}\<[5]%
\>[5]{}\nonterm{\AF{unload}}\;\mathbin{:}\;\AD{\ensuremath{\mathbb{N}}}\;\to \;\AD{Stack}\;\to \;\AD{\ensuremath{\mathbb{N}}}{}\<[E]%
\\
\>[3]{}\hsindent{2}{}\<[5]%
\>[5]{}\AF{unload}\;\AB{v}\;{}\<[16]%
\>[16]{}\AI{Top}\;{}\<[32]%
\>[32]{}\mathrel{=}\;\AB{v}{}\<[E]%
\\
\>[3]{}\hsindent{2}{}\<[5]%
\>[5]{}\AF{unload}\;\AB{v}\;{}\<[16]%
\>[16]{}(\AI{Right}\;\AB{v'}\;\AB{stk})\;{}\<[32]%
\>[32]{}\mathrel{=}\;\nonterm{\AF{unload}}\;(\AB{v'}\;\AF{+}\;\AB{v})\;\AB{stk}{}\<[E]%
\\
\>[3]{}\hsindent{2}{}\<[5]%
\>[5]{}\AF{unload}\;\AB{v}\;{}\<[16]%
\>[16]{}(\AI{Left}\;\AB{r}\;\AB{stk})\;{}\<[32]%
\>[32]{}\mathrel{=}\;\nonterm{\AF{load}}\;\AB{r}\;(\AI{Right}\;\AB{v}\;\AB{stk}){}\<[E]%
\ColumnHook
\end{hscode}\resethooks

We can now define a tail recursive version of \ensuremath{\AF{eval}} by
calling \ensuremath{\AF{load}} with an initially empty stack:

\begin{hscode}\SaveRestoreHook
\column{B}{@{}>{\hspre}l<{\hspost}@{}}%
\column{3}{@{}>{\hspre}l<{\hspost}@{}}%
\column{E}{@{}>{\hspre}l<{\hspost}@{}}%
\>[3]{}\AF{tail\text{-}rec\text{-}eval}\;\mathbin{:}\;\AD{Expr}\;\Varid{→}\;\AD{\ensuremath{\mathbb{N}}}{}\<[E]%
\\
\>[3]{}\AF{tail\text{-}rec\text{-}eval}\;\AB{e}\;\mathrel{=}\;\AF{load}\;\AB{e}\;\AI{Top}{}\<[E]%
\ColumnHook
\end{hscode}\resethooks

Implementing this tail recursive evaluator has come at a price: Agda's
termination checker flags the \ensuremath{\AF{load}} and \ensuremath{\AF{unload}} functions as
potentially non-terminating by highlighting them
\nonterm{orange}. Indeed, in the very last clause of the \ensuremath{\AF{unload}}
function a recursive call is made to arguments that are not
syntactically smaller. Furthermore, it is not clear at all that the
tail recursive evaluator produces the same result as our original
one. It is precisely these questions that this paper tackles by
means of the following novel contributions:

\begin{itemize}
\item We give a verified proof of termination of \ensuremath{\AF{tail\text{-}rec\text{-}eval}} using
  a carefully chosen \emph{well-founded relation}
  (Section~\ref{sec:basics}). After redefining \ensuremath{\AF{tail\text{-}rec\text{-}eval}} using
  this relation, we prove that it agrees with the \ensuremath{\AF{eval}} function on
  all inputs.
\item We generalize this construction, inspired by McBride's
  generalization of Danvy's work on abstract machines. To do so, we
  define a universe of algebraic data types and a generic fold
  (Section~\ref{sec:universe}). Subsequently we show how to turn any
  structurally recursive function defined using a fold into a tail
  recursive variant (Section~\ref{sec:dissection}). 
\item Finally, we sketch how the proofs of termination and semantics
  preservation from our example are generalized to the generic fold
  over arbitrary types in our universe (Section~\ref{correctness}).
\end{itemize}

All the constructions and proofs presented in this paper have been
implemented in Agda\todo{citation}. The corresponding code is freely
available online.\footnote{\todo{url}}

\section{Basic idea}
\label{sec:basics}
The functions \ensuremath{\AF{load}} and \ensuremath{\AF{unload}} are marked as non terminating
because they are not defined by structural recursion over their
arguments. In particular, the stack passed as an argument to the
recursive call of \ensuremath{\AF{load}} in the definition of \ensuremath{\AF{unload}} is structurally
equal in size as the input stack.

Intuitively, \ensuremath{\AF{load}} and \ensuremath{\AF{unload}} fold the tree by traversing it from
its leftmost leaf to its rightmost using the stack to store both
partial results and the remaining subtrees to fold them as
neccesary. The problem arises because the stack is simply typed and
any information about how the subtrees kept in the stack relate to
each other and to the original tree is lost once a subtree is inserted
onto the stack.

However, it is clear that virtually every node (either leaf or not)
from the original tree is visited at most twice during the
computation. First when the function \ensuremath{\AF{load}} decomposes it looking for
its leftmost leaf and a second time when \ensuremath{\AF{unload}} is accumulating over
the stack searching for another subtree to continue. This process is
depicted in figure 1.

\begin{figure}[h]
  \includegraphics[scale=0.25]{figure1}
\end{figure}

We can argue that because there are finitely many nodes on a tree,
\ensuremath{\AF{load}} and \ensuremath{\AF{unload}} neccesarily terminate. The question is now, How
can we encode this information in such a way that Agda understand that
the fold terminates?

The idea is that \ensuremath{\AF{load}} and \ensuremath{\AF{unload}} should not fold the full input
tree in one go, but instead they will perform one step of the
computation at a time.  Morover, by defining them by structural
recursion over their arguments now they are classified as terminating
by the termination checker.

\begin{hscode}\SaveRestoreHook
\column{B}{@{}>{\hspre}l<{\hspost}@{}}%
\column{3}{@{}>{\hspre}l<{\hspost}@{}}%
\column{14}{@{}>{\hspre}l<{\hspost}@{}}%
\column{21}{@{}>{\hspre}l<{\hspost}@{}}%
\column{30}{@{}>{\hspre}l<{\hspost}@{}}%
\column{E}{@{}>{\hspre}l<{\hspost}@{}}%
\>[3]{}\AF{load}\;\mathbin{:}\;\AD{Expr}\;\to \;\AD{Stack}\;\to \;\AD{\ensuremath{\mathbb{N}}}\;\AF{\ensuremath{\times}}\;\AD{Stack}{}\<[E]%
\\
\>[3]{}\AF{load}\;(\AI{Val}\;\AB{n})\;{}\<[21]%
\>[21]{}\AB{stk}\;\mathrel{=}\;(\AB{n}\;\Varid{,}\;\AB{stk}){}\<[E]%
\\
\>[3]{}\AF{load}\;(\AI{Add}\;\AB{\ensuremath{e_1}}\;\AB{\ensuremath{e_2}})\;{}\<[21]%
\>[21]{}\AB{stk}\;\mathrel{=}\;\AF{load}\;\AB{\ensuremath{e_1}}\;(\AI{Left}\;\AB{\ensuremath{e_2}}\;\AB{stk}){}\<[E]%
\\[\blanklineskip]%
\>[3]{}\AF{unload}\;\mathbin{:}\;\AD{\ensuremath{\mathbb{N}}}\;\to \;\AD{Stack}\;\to \;(\AD{\ensuremath{\mathbb{N}}}\;\AF{\ensuremath{\times}}\;\AD{Stack})\;\AD{\ensuremath{\uplus}}\;\AD{\ensuremath{\mathbb{N}}}{}\<[E]%
\\
\>[3]{}\AF{unload}\;\AB{v}\;{}\<[14]%
\>[14]{}\AI{Top}\;{}\<[30]%
\>[30]{}\mathrel{=}\;\AI{inj\ensuremath{_2}}\;\AB{v}{}\<[E]%
\\
\>[3]{}\AF{unload}\;\AB{v}\;{}\<[14]%
\>[14]{}(\AI{Right}\;\AB{v'}\;\AB{stk})\;{}\<[30]%
\>[30]{}\mathrel{=}\;\AF{unload}\;(\AB{v'}\;\AF{+}\;\AB{v})\;\AB{stk}{}\<[E]%
\\
\>[3]{}\AF{unload}\;\AB{v}\;{}\<[14]%
\>[14]{}(\AI{Left}\;\AB{r}\;\AB{stk})\;{}\<[30]%
\>[30]{}\mathrel{=}\;\AI{inj\ensuremath{_1}}\;(\AF{load}\;\AB{r}\;(\AI{Right}\;\AB{v}\;\AB{stk})){}\<[E]%
\ColumnHook
\end{hscode}\resethooks

For example, if we take the same tree in figure 1, after \ensuremath{\AF{load}} finds
the initial leftmost leaf we can apply one step of the new \ensuremath{\AF{unload}}
that will end up in the next leaf to the right.


\begin{figure}[h]
  \includegraphics[scale=0.25]{figure2}
\end{figure}

A tail recursive fold corrensponds to repeatedly applying the function
\ensuremath{\AF{unload}} until we find a \ensuremath{\AI{inj\ensuremath{_2}}} whose value is the result of folding
the tree.

\begin{hscode}\SaveRestoreHook
\column{B}{@{}>{\hspre}l<{\hspost}@{}}%
\column{3}{@{}>{\hspre}l<{\hspost}@{}}%
\column{5}{@{}>{\hspre}l<{\hspost}@{}}%
\column{7}{@{}>{\hspre}l<{\hspost}@{}}%
\column{E}{@{}>{\hspre}l<{\hspost}@{}}%
\>[3]{}\AF{tail\text{-}rec\text{-}eval}\;\mathbin{:}\;\AD{Expr}\;\to \;\AD{\ensuremath{\mathbb{N}}}{}\<[E]%
\\
\>[3]{}\AF{tail\text{-}rec\text{-}eval}\;\AB{e}\;\mathrel{=}\;\AF{rec}\;(\AF{load}\;\AB{e}\;\AI{Top}){}\<[E]%
\\
\>[3]{}\hsindent{2}{}\<[5]%
\>[5]{}\AK{where}{}\<[E]%
\\
\>[5]{}\hsindent{2}{}\<[7]%
\>[7]{}\nonterm{\AF{rec}}\;\mathbin{:}\;(\AD{\ensuremath{\mathbb{N}}}\;\AF{\ensuremath{\times}}\;\AD{Stack})\;\to \;\AD{\ensuremath{\mathbb{N}}}{}\<[E]%
\\
\>[5]{}\hsindent{2}{}\<[7]%
\>[7]{}\AF{rec}\;(\AB{n}\;\Varid{,}\;\AB{stk})\;\AK{with}\;\AF{unload}\;\AB{n}\;\AB{stk}{}\<[E]%
\\
\>[5]{}\hsindent{2}{}\<[7]%
\>[7]{}\Varid{...}\;\mid \;\AI{inj\ensuremath{_1}}\;\AB{z'}\;\mathrel{=}\;\nonterm{\AF{rec}}\;\AB{z'}{}\<[E]%
\\
\>[5]{}\hsindent{2}{}\<[7]%
\>[7]{}\Varid{...}\;\mid \;\AI{inj\ensuremath{_2}}\;\AB{r}\;\mathrel{=}\;\AB{r}{}\<[E]%
\ColumnHook
\end{hscode}\resethooks

The function \ensuremath{\AF{tail\text{-}rec\text{-}eval}} still does not pass the termination
checker, The variable \ensuremath{\AB{z'}} is not structurally smaller than \ensuremath{(\AB{n}\;\Varid{,}\;\AB{stk})}. However, now we can refine it by using well founded recursion
to make it structurally recursive by performing the recursion over the
accessibility predicate instead of the pair \ensuremath{\AD{\ensuremath{\mathbb{N}}}\;\AF{\ensuremath{\times}}\;\AD{Stack}}.

\begin{hscode}\SaveRestoreHook
\column{B}{@{}>{\hspre}l<{\hspost}@{}}%
\column{3}{@{}>{\hspre}l<{\hspost}@{}}%
\column{5}{@{}>{\hspre}l<{\hspost}@{}}%
\column{7}{@{}>{\hspre}l<{\hspost}@{}}%
\column{E}{@{}>{\hspre}l<{\hspost}@{}}%
\>[3]{}\AF{tail\text{-}rec\text{-}eval}\;\mathbin{:}\;\AD{Expr}\;\to \;\AD{\ensuremath{\mathbb{N}}}{}\<[E]%
\\
\>[3]{}\AF{tail\text{-}rec\text{-}eval}\;\AB{e}\;\mathrel{=}\;\AF{rec}\;(\AF{load}\;\AB{e}\;\AI{Top})\;\hole{1}{}\<[E]%
\\
\>[3]{}\hsindent{2}{}\<[5]%
\>[5]{}\AK{where}{}\<[E]%
\\
\>[5]{}\hsindent{2}{}\<[7]%
\>[7]{}\AF{rec}\;\mathbin{:}\;(\AB{z}\;\mathbin{:}\;\AD{\ensuremath{\mathbb{N}}}\;\AF{\ensuremath{\times}}\;\AD{Stack})\;\to \;\AD{Acc}\;\AD{\_<\_}\;\AB{z}\;\to \;\AD{\ensuremath{\mathbb{N}}}{}\<[E]%
\\
\>[5]{}\hsindent{2}{}\<[7]%
\>[7]{}\AF{rec}\;(\AB{n}\;\Varid{,}\;\AB{stk})\;(\AI{acc}\;\Varid{rs})\;\AK{with}\;\AF{unload}\;\AB{n}\;\AB{stk}{}\<[E]%
\\
\>[5]{}\hsindent{2}{}\<[7]%
\>[7]{}\Varid{...}\;\mid \;\AI{inj\ensuremath{_1}}\;\AB{z'}\;\mathrel{=}\;\AF{rec}\;\AB{z'}\;(\Varid{rs}\;\hole{2}){}\<[E]%
\\
\>[5]{}\hsindent{2}{}\<[7]%
\>[7]{}\Varid{...}\;\mid \;\AI{inj\ensuremath{_2}}\;\AB{r}\;\mathrel{=}\;\AB{r}{}\<[E]%
\ColumnHook
\end{hscode}\resethooks

For the function above to work, we need to find a suitable definition
for the relation \ensuremath{\AD{\_<\_}} over pairs of \ensuremath{\AD{\ensuremath{\mathbb{N}}}\;\AF{\ensuremath{\times}}\;\AD{Stack}}, prove that
applying \ensuremath{\AF{unload}} results in an smaller element by the relation and
finally show that the relation is \ensuremath{\AF{Well\text{-}founded}}, so the call to \ensuremath{\AF{rec}}
can be made in the first place. Before any of that, we need to revisit
Huet's notion of \emph{Zipper} and show how it relates to what we are
trying to achieve.

\subsection{Zippers up, Zippers down}

Huet introduced \emph{Zippers} to allow a tree datastructure to be
efficiently updated in a purely functional way. The idea is that any
location on a tree, either an internal node or a leaf, can be
represented by a path to the root and the subtree that hangs
downwards.

The pair \ensuremath{\AD{\ensuremath{\mathbb{N}}}\;\AF{\ensuremath{\times}}\;\AD{Stack}} used to compute the tail recursive fold is
nothing more that a restricted version of the \emph{Zipper} where the
locations can only be leaves of the tree.

\begin{hscode}\SaveRestoreHook
\column{B}{@{}>{\hspre}l<{\hspost}@{}}%
\column{3}{@{}>{\hspre}l<{\hspost}@{}}%
\column{E}{@{}>{\hspre}l<{\hspost}@{}}%
\>[3]{}\AD{Zipper}\;\mathbin{:}\;\AP{Set}{}\<[E]%
\\
\>[3]{}\AD{Zipper}\;\mathrel{=}\;\AD{\ensuremath{\mathbb{N}}}\;\AF{\ensuremath{\times}}\;\AD{Stack}{}\<[E]%
\ColumnHook
\end{hscode}\resethooks

From a \ensuremath{\AD{Zipper}} we have to be able to reconstruct the original \ensuremath{\AD{Expr}}
which will be neccesary later on for the proof that the relation is
well founded. For this matter, we have to enhance the type of stacks
to store not only the partial results but also the expressions that
where consumed in order to produce them.

\begin{hscode}\SaveRestoreHook
\column{B}{@{}>{\hspre}l<{\hspost}@{}}%
\column{3}{@{}>{\hspre}l<{\hspost}@{}}%
\column{25}{@{}>{\hspre}l<{\hspost}@{}}%
\column{26}{@{}>{\hspre}l<{\hspost}@{}}%
\column{27}{@{}>{\hspre}l<{\hspost}@{}}%
\column{E}{@{}>{\hspre}l<{\hspost}@{}}%
\>[3]{}\AD{Stack}\;\mathbin{:}\;\AP{Set}{}\<[E]%
\\
\>[3]{}\AD{Stack}\;\mathrel{=}\;\AD{List}\;(\AD{Expr}\;\AD{\ensuremath{\uplus}}\;{}\<[26]%
\>[26]{}(\AD{\ensuremath{\Sigma}}\;\AD{\ensuremath{\mathbb{N}}}\;\AK{\ensuremath{\lambda}}\;\AB{n}\;\to {}\<[E]%
\\
\>[26]{}\hsindent{1}{}\<[27]%
\>[27]{}\AD{\ensuremath{\Sigma}}\;\AD{Expr}\;\AK{\ensuremath{\lambda}}\;\AB{e}\;\to \;\AF{eval}\;\AB{e}\;\AD{\ensuremath{\equiv}}\;\AB{n})){}\<[E]%
\\
\>[3]{}\AK{pattern}\;\AI{Left}\;\AB{\ensuremath{t}}\;{}\<[25]%
\>[25]{}\mathrel{=}\;\AI{inj\ensuremath{_1}}\;\AB{\ensuremath{t}}{}\<[E]%
\\
\>[3]{}\AK{pattern}\;\AI{Right}\;\AB{n}\;\AB{\ensuremath{t}}\;\AB{\ensuremath{eq}}\;{}\<[25]%
\>[25]{}\mathrel{=}\;\AI{inj\ensuremath{_2}}\;(\AB{n}\;\Varid{,}\;\AB{\ensuremath{t}}\;\Varid{,}\;\AB{\ensuremath{eq}}){}\<[E]%
\ColumnHook
\end{hscode}\resethooks

The original expression for which a \ensuremath{\AD{\ensuremath{\mathbb{N}}}\;\AF{\ensuremath{\times}}\;\AD{Stack}} represents a
position can be reconstructed by forgeting that some part has already
been evaluated.

\begin{hscode}\SaveRestoreHook
\column{B}{@{}>{\hspre}l<{\hspost}@{}}%
\column{3}{@{}>{\hspre}l<{\hspost}@{}}%
\column{27}{@{}>{\hspre}l<{\hspost}@{}}%
\column{36}{@{}>{\hspre}l<{\hspost}@{}}%
\column{E}{@{}>{\hspre}l<{\hspost}@{}}%
\>[3]{}\AF{plug\ensuremath{\Uparrow}}\;\mathbin{:}\;\AD{Expr}\;\to \;\AD{Stack}\;\to \;\AD{Expr}{}\<[E]%
\\
\>[3]{}\AF{plug\ensuremath{\Uparrow}}\;\AB{e}\;\AI{[]}\;{}\<[36]%
\>[36]{}\mathrel{=}\;\AB{e}{}\<[E]%
\\
\>[3]{}\AF{plug\ensuremath{\Uparrow}}\;\AB{e}\;(\AI{Left}\;\AB{\ensuremath{t}}\;{}\<[27]%
\>[27]{}\hspace{.4em}\AI{:\hspace{-.1em}:}\hspace{.4em}\;\AB{stk})\;{}\<[36]%
\>[36]{}\mathrel{=}\;\AF{plug\ensuremath{\Uparrow}}\;(\AI{Add}\;\AB{e}\;\AB{\ensuremath{t}})\;\AB{stk}{}\<[E]%
\\
\>[3]{}\AF{plug\ensuremath{\Uparrow}}\;\AB{e}\;(\AI{Right}\;\AB{n}\;\AB{\ensuremath{t}}\;\AB{\ensuremath{eq}}\;{}\<[27]%
\>[27]{}\hspace{.4em}\AI{:\hspace{-.1em}:}\hspace{.4em}\;\AB{stk})\;{}\<[36]%
\>[36]{}\mathrel{=}\;\AF{plug\ensuremath{\Uparrow}}\;(\AI{Add}\;\AB{\ensuremath{t}}\;\AB{e})\;\AB{stk}{}\<[E]%
\\[\blanklineskip]%
\>[3]{}\AF{plugZ\ensuremath{\Uparrow}}\;\mathbin{:}\;(\AD{\ensuremath{\mathbb{N}}}\;\AF{\ensuremath{\times}}\;\AD{Stack})\;\to \;\AD{Expr}{}\<[E]%
\\
\>[3]{}\AF{plugZ\ensuremath{\Uparrow}}\;(\AB{n}\;\Varid{,}\;\AB{stk})\;\mathrel{=}\;\AF{plug\ensuremath{\Uparrow}}\;(\AI{Val}\;\AB{n})\;\AB{stk}{}\<[E]%
\ColumnHook
\end{hscode}\resethooks

Our goal is to impose an ordering relation over elements of \ensuremath{\AD{Zipper}}
such that for any input the \ensuremath{\AF{unload}} function delivers a \ensuremath{\AD{Zipper}} that
is smaller by the relation when it does not terminate with a
value. Because the folding happens from left to right, we want the
relation to order the leaves of the tree accordingly, so the leftmost
leaf is the greatest element and the rightmost the smallest. Using the
example from before, we number the leaves as follows:

\begin{figure}[h]
  \includegraphics[scale=0.25, angle=90]{figure3}
\end{figure}

Using the \ensuremath{\AD{Stack}} as a path from the leaf to the root of the tree is
difficult if not impossible to encode a smaller than relation that
does not relate any two elements. Such relation has to be defined by
induction on the \ensuremath{\AD{Stack}} part of the \ensuremath{\AD{Zipper}}. But for any two given
stacks a priori we cannot know how many layers we have to peel in
order to reach a case where one of them is obviously smaller that the
other.

We can approach the problem by understanding the \ensuremath{\AD{Stack}} not as a path
from the leaf up to the root but from the root down to the leaf. This
change of perspective is realised with a new plug function that does
the opposite of \ensuremath{\AF{plug\ensuremath{\Uparrow}}}.

\begin{hscode}\SaveRestoreHook
\column{B}{@{}>{\hspre}l<{\hspost}@{}}%
\column{3}{@{}>{\hspre}l<{\hspost}@{}}%
\column{28}{@{}>{\hspre}l<{\hspost}@{}}%
\column{37}{@{}>{\hspre}l<{\hspost}@{}}%
\column{E}{@{}>{\hspre}l<{\hspost}@{}}%
\>[3]{}\AF{plug\ensuremath{\Downarrow}}\;\mathbin{:}\;\AD{Expr}\;\to \;\AD{Stack}\;\to \;\AD{Expr}{}\<[E]%
\\
\>[3]{}\AF{plug\ensuremath{\Downarrow}}\;\AB{e}\;\AI{[]}\;{}\<[37]%
\>[37]{}\mathrel{=}\;\AB{e}{}\<[E]%
\\
\>[3]{}\AF{plug\ensuremath{\Downarrow}}\;\AB{e}\;(\AI{Left}\;\AB{\ensuremath{t}}\;{}\<[28]%
\>[28]{}\hspace{.4em}\AI{:\hspace{-.1em}:}\hspace{.4em}\;\AB{stk})\;{}\<[37]%
\>[37]{}\mathrel{=}\;\AI{Add}\;(\AF{plug\ensuremath{\Downarrow}}\;\AB{e}\;\AB{stk})\;\AB{\ensuremath{t}}{}\<[E]%
\\
\>[3]{}\AF{plug\ensuremath{\Downarrow}}\;\AB{e}\;(\AI{Right}\;\AB{n}\;\anonymous \;\anonymous \;{}\<[28]%
\>[28]{}\hspace{.4em}\AI{:\hspace{-.1em}:}\hspace{.4em}\;\AB{stk})\;{}\<[37]%
\>[37]{}\mathrel{=}\;\AI{Add}\;\AB{\ensuremath{t}}\;(\AF{plug\ensuremath{\Downarrow}}\;\AB{e}\;\AB{stk}){}\<[E]%
\\[\blanklineskip]%
\>[3]{}\AF{plugZ\ensuremath{\Downarrow}}\;\mathbin{:}\;(\AD{\ensuremath{\mathbb{N}}}\;\AF{\ensuremath{\times}}\;\AD{Stack})\;\to \;\AD{Expr}{}\<[E]%
\\
\>[3]{}\AF{plugZ\ensuremath{\Downarrow}}\;(\AB{n}\;\Varid{,}\;\AB{stk})\;\mathrel{=}\;\AF{plug\ensuremath{\Downarrow}}\;(\AI{Val}\;\AB{n})\;\AB{stk}{}\<[E]%
\ColumnHook
\end{hscode}\resethooks

It is clear that both views of the \ensuremath{\AD{Zipper}} are related. Indeed, to
transport from one to the other we only have to reverse the stack. We
show the equvalence with the following lemma\footnote{The other way
  around only requires to use BLABLA of \ensuremath{\AF{reverse}}}:

\begin{hscode}\SaveRestoreHook
\column{B}{@{}>{\hspre}l<{\hspost}@{}}%
\column{3}{@{}>{\hspre}l<{\hspost}@{}}%
\column{23}{@{}>{\hspre}l<{\hspost}@{}}%
\column{44}{@{}>{\hspre}l<{\hspost}@{}}%
\column{E}{@{}>{\hspre}l<{\hspost}@{}}%
\>[3]{}\AF{plug\ensuremath{\Downarrow}}\AF{\text{-}to\text{-}}\AF{plug\ensuremath{\Uparrow}}\;{}\<[23]%
\>[23]{}\mathbin{:}\;\forall\;(\AB{e}\;\mathbin{:}\;\AD{Expr})\;(\AB{stk}\;\mathbin{:}\;\AD{Stack}){}\<[E]%
\\
\>[23]{}\Varid{→}\;\AF{plug\ensuremath{\Downarrow}}\;\AB{e}\;\AB{stk}\;\AD{\ensuremath{\equiv}}\;{}\<[44]%
\>[44]{}\AF{plug\ensuremath{\Uparrow}}\;\AB{e}\;(\AF{reverse}\;\AB{stk}){}\<[E]%
\ColumnHook
\end{hscode}\resethooks


Why do we need this equivalence? The bottom up view of a \ensuremath{\AD{Zipper}} is
suitable for defining the tail recursive fold, alas to prove
termination we have to use use the top down view to describe the
relation we need.

\subsection{A relation on Zipper}

The relation over elements of \ensuremath{\AD{Zipper}} is defined by induction on the
\ensuremath{\AD{Stack}}.  If we start in the root of the tree, we can navigate
downwards both stacks in parallel removing their common prefix. Once
we find an \ensuremath{\AI{Add}} where they disagree then whether the first \ensuremath{\AD{Zipper}}
is located in the left or right subtree fully determines if its bigger
or smaller than the other \ensuremath{\AD{Zipper}}.  The following type accounts for
this explanation:

\begin{hscode}\SaveRestoreHook
\column{B}{@{}>{\hspre}l<{\hspost}@{}}%
\column{3}{@{}>{\hspre}l<{\hspost}@{}}%
\column{5}{@{}>{\hspre}l<{\hspost}@{}}%
\column{14}{@{}>{\hspre}l<{\hspost}@{}}%
\column{18}{@{}>{\hspre}l<{\hspost}@{}}%
\column{19}{@{}>{\hspre}l<{\hspost}@{}}%
\column{23}{@{}>{\hspre}l<{\hspost}@{}}%
\column{44}{@{}>{\hspre}l<{\hspost}@{}}%
\column{E}{@{}>{\hspre}l<{\hspost}@{}}%
\>[3]{}\AK{data}\;\AD{\_<\_}\;\mathbin{:}\;\AD{Zipper}\;\to \;\AD{Zipper}\;\to \;\AP{Set}\;\AK{where}{}\<[E]%
\\
\>[3]{}\hsindent{2}{}\<[5]%
\>[5]{}\AI{<\text{-}Right}\;{}\<[14]%
\>[14]{}\mathbin{:}\;(\AB{\ensuremath{t_1}}\;\Varid{,}\;\AB{\ensuremath{s_1}})\;\AD{<}\;(\AB{\ensuremath{t_2}}\;\Varid{,}\;\AB{\ensuremath{s_2}})\;{}\<[E]%
\\
\>[14]{}\to \;{}\<[18]%
\>[18]{}(\AB{\ensuremath{t_1}}\;\Varid{,}\;\AI{Right}\;\Varid{l}\;\AB{n}\;\AB{\ensuremath{eq}}\;\hspace{.4em}\AI{:\hspace{-.1em}:}\hspace{.4em}\;\AB{\ensuremath{s_1}})\;\AD{<}\;(\AB{\ensuremath{t_2}}\;\Varid{,}\;\AI{Right}\;\Varid{l}\;\AB{n}\;\AB{\ensuremath{eq}}\;\hspace{.4em}\AI{:\hspace{-.1em}:}\hspace{.4em}\;\AB{\ensuremath{s_2}}){}\<[E]%
\\
\>[3]{}\hsindent{2}{}\<[5]%
\>[5]{}\AI{<\text{-}Left}\;{}\<[14]%
\>[14]{}\mathbin{:}\;(\AB{\ensuremath{t_1}}\;\Varid{,}\;\AB{\ensuremath{s_1}})\;\AD{<}\;(\AB{\ensuremath{t_2}}\;\Varid{,}\;\AB{\ensuremath{s_2}})\;{}\<[E]%
\\
\>[14]{}\to \;{}\<[18]%
\>[18]{}(\AB{\ensuremath{t_1}}\;\Varid{,}\;\AI{Left}\;\AB{r}\;\hspace{.4em}\AI{:\hspace{-.1em}:}\hspace{.4em}\;\AB{\ensuremath{s_1}})\;{}\<[44]%
\>[44]{}\AD{<}\;(\AB{\ensuremath{t_2}}\;\Varid{,}\;\AI{Left}\;\AB{r}\;\hspace{.4em}\AI{:\hspace{-.1em}:}\hspace{.4em}\;\AB{\ensuremath{s_2}}){}\<[E]%
\\
\>[3]{}\hsindent{2}{}\<[5]%
\>[5]{}\AI{<\text{-}Right\text{-}Left}\;{}\<[19]%
\>[19]{}\mathbin{:}\;{}\<[23]%
\>[23]{}(\AB{\ensuremath{t_1}'}\;\AD{\ensuremath{\equiv}}\;\AF{plug\ensuremath{\Downarrow}}\;(\Conid{Tip}\;\AB{\ensuremath{t_2}})\;\AB{\ensuremath{s_2}})\;{}\<[E]%
\\
\>[19]{}\to \;{}\<[23]%
\>[23]{}(\AB{\ensuremath{t_2}'}\;\AD{\ensuremath{\equiv}}\;\AF{plug\ensuremath{\Downarrow}}\;(\Conid{Tip}\;\AB{\ensuremath{t_1}})\;\AB{\ensuremath{s_1}})\;{}\<[E]%
\\
\>[19]{}\to \;{}\<[23]%
\>[23]{}(\AB{\ensuremath{t_1}}\;\Varid{,}\;\AI{Right}\;\AB{n}\;\AB{\ensuremath{t_1}'}\;\AB{\ensuremath{eq}}\;\hspace{.4em}\AI{:\hspace{-.1em}:}\hspace{.4em}\;\AB{\ensuremath{s_1}})\;\AD{<}\;(\AB{\ensuremath{t_2}}\;\Varid{,}\;\AI{Left}\;\AB{\ensuremath{t_2}'}\;\hspace{.4em}\AI{:\hspace{-.1em}:}\hspace{.4em}\;\AB{\ensuremath{s_2}}){}\<[E]%
\ColumnHook
\end{hscode}\resethooks

Having the relation defined, we turn our focus to prove that it is
well founded.  This is an important step towards filling the holes
that were left open in the function \ensuremath{\AF{tail\text{-}rec\text{-}eval}}.

A relation is well founded iff all the descending chains starting from
an arbitrary element are finite. In a theorem prover such as Agda, an
alternative definition of well foundedness is used which is based on
an accesibility predicate.

We can try to prove that the relation is well founded by using an
auxiliary function that allows us to pattern match on the smaller than
proof. When doing so, the inputs are refined to concrete
constructors. Normally the proof either makes use of recursion over
the proof or over the input, but in the case of the \ensuremath{\AI{<\text{-}Right\text{-}Left}}
constructor we do not have either option, because the smaller element
is not structurally related to the bigger and the proof does not have
any recursive structure to use.

\begin{hscode}\SaveRestoreHook
\column{B}{@{}>{\hspre}l<{\hspost}@{}}%
\column{5}{@{}>{\hspre}l<{\hspost}@{}}%
\column{7}{@{}>{\hspre}l<{\hspost}@{}}%
\column{11}{@{}>{\hspre}l<{\hspost}@{}}%
\column{E}{@{}>{\hspre}l<{\hspost}@{}}%
\>[B]{}\Varid{<-WF}\;\mathbin{:}\;\AF{Well\text{-}founded}\;\AD{\_<\_}{}\<[E]%
\\
\>[B]{}\Varid{<-WF}\;\Varid{x}\;\mathrel{=}\;\AI{acc}\;(\Varid{aux}\;\Varid{x}){}\<[E]%
\\
\>[B]{}\hsindent{5}{}\<[5]%
\>[5]{}\AK{where}{}\<[E]%
\\
\>[5]{}\hsindent{2}{}\<[7]%
\>[7]{}\Varid{aux}\;\mathbin{:}\;\Varid{∀}\;(\Varid{x}\;\mathbin{:}\;\AD{Zipper})\;{}\<[E]%
\\
\>[7]{}\hsindent{4}{}\<[11]%
\>[11]{}\to \;\Varid{∀}\;(\Varid{y}\;\mathbin{:}\;\AD{Zipper})\;\to \;\Varid{y}\;\AD{<}\;\Varid{x}\;\to \;\AD{Acc}\;\AD{\_<\_}\;\Varid{y}{}\<[E]%
\\
\>[5]{}\hsindent{2}{}\<[7]%
\>[7]{}\Varid{aux}\;\AS{.}\hspace*{-0.1cm}\;(\AB{\ensuremath{t_2}}\;\Varid{,}\;\AI{Left}\;\AB{\ensuremath{t_1}'}\;\hspace{.4em}\AI{:\hspace{-.1em}:}\hspace{.4em}\;\AB{\ensuremath{s_2}})\;\AS{.}\hspace*{-0.1cm}\;(\AB{\ensuremath{t_1}}\;\Varid{,}\;\AI{Right}\;\AB{n}\;\AB{\ensuremath{t_2}'}\;\AB{\ensuremath{eq}}\;\hspace{.4em}\AI{:\hspace{-.1em}:}\hspace{.4em}\;\AB{\ensuremath{s_1}})\;(\AI{<\text{-}Right\text{-}Left}\;\Varid{eq1}\;\Varid{eq2})\;\mathrel{=}\;\{\mskip1.5mu \mathbin{!!}\mskip1.5mu\}{}\<[E]%
\\
\>[5]{}\hsindent{2}{}\<[7]%
\>[7]{}\Varid{aux}\;\Varid{...}{}\<[E]%
\ColumnHook
\end{hscode}\resethooks

The proof fails because in \ensuremath{\Varid{aux}} both Zippers \ensuremath{\Varid{x}} and \ensuremath{\Varid{y}} might very
well be locations of leaves belonging to different trees as far we
know. Thanks to the use of dependent types, the property that a Zipper
represents a position inside a concrete tree can be made explicit at
the type level.

\begin{hscode}\SaveRestoreHook
\column{B}{@{}>{\hspre}l<{\hspost}@{}}%
\column{3}{@{}>{\hspre}l<{\hspost}@{}}%
\column{5}{@{}>{\hspre}l<{\hspost}@{}}%
\column{E}{@{}>{\hspre}l<{\hspost}@{}}%
\>[3]{}\AK{data}\;\AD{Zipper\ensuremath{\Downarrow}}\;(\AB{e}\;\mathbin{:}\;\AD{Expr})\;\mathbin{:}\;\AP{Set}\;\AK{where}{}\<[E]%
\\
\>[3]{}\hsindent{2}{}\<[5]%
\>[5]{}\Varid{\char92 \char95 ,\char92 \char95 }\;\mathbin{:}\;(\AB{z}\;\mathbin{:}\;\AD{Zipper})\;\to \;\AF{plugZ\ensuremath{\Downarrow}}\;\AB{z}\;\AD{\ensuremath{\equiv}}\;\AB{e}\;\to \;\AD{Zipper\ensuremath{\Downarrow}}\;\AB{e}{}\<[E]%
\ColumnHook
\end{hscode}\resethooks

We write a relation that is enforced to only relate Zippers beloging
to the same \ensuremath{\AD{Expr}} by using a common value of that type as an of
\ensuremath{\AD{Zipper\ensuremath{\Downarrow}}}

\begin{hscode}\SaveRestoreHook
\column{B}{@{}>{\hspre}l<{\hspost}@{}}%
\column{3}{@{}>{\hspre}l<{\hspost}@{}}%
\column{5}{@{}>{\hspre}l<{\hspost}@{}}%
\column{E}{@{}>{\hspre}l<{\hspost}@{}}%
\>[3]{}\AK{data}\;\Conid{IxltOp}\;\mathbin{:}\;(\AB{e}\;\mathbin{:}\;\AD{Expr})\;\to \;\AD{Zipper\ensuremath{\Downarrow}}\;\AB{e}\;\to \;\AD{Zipper\ensuremath{\Downarrow}}\;\AB{e}\;\to \;\AP{Set}\;\AK{where}{}\<[E]%
\\
\>[3]{}\hsindent{2}{}\<[5]%
\>[5]{}\Varid{...}{}\<[E]%
\ColumnHook
\end{hscode}\resethooks

The concrete details of the relation follow very much the one we gave
before, with the exception that every case has attached a new piece of
information specifying the concrete \ensuremath{\AD{Expr}} obtained by plugging both
Zippers.

The new version of the relation is suitable for proving well
foundedness because we can pattern match on the equality included in
the \ensuremath{\AD{Zipper\ensuremath{\Downarrow}}} type to show how the overall structure
decreases. This allows us to use the recursion we need to complete the
proof.  In particular, the case we were not able to prove before, now
can be proven by learning that \ensuremath{(\AB{\ensuremath{t_2}}\;\Varid{,}\;\AB{\ensuremath{s_2}})} is a position on the left
subtree while \ensuremath{(\AB{\ensuremath{t_1}}\;\Varid{,}\;\AB{\ensuremath{s_1}})} is on the right subtree of a common \ensuremath{\AI{Add}}
node.

\begin{hscode}\SaveRestoreHook
\column{B}{@{}>{\hspre}l<{\hspost}@{}}%
\column{3}{@{}>{\hspre}l<{\hspost}@{}}%
\column{5}{@{}>{\hspre}l<{\hspost}@{}}%
\column{7}{@{}>{\hspre}l<{\hspost}@{}}%
\column{11}{@{}>{\hspre}l<{\hspost}@{}}%
\column{E}{@{}>{\hspre}l<{\hspost}@{}}%
\>[3]{}\Varid{<-WF}\;\mathbin{:}\;\forall\;\AB{e}\;\to \;\AF{Well\text{-}founded}\;(\Conid{IxltOp}\;\AB{e}){}\<[E]%
\\
\>[3]{}\Varid{<-WF}\;\Varid{x}\;\mathrel{=}\;\AI{acc}\;(\Varid{aux}\;\AB{e}\;\Varid{x}){}\<[E]%
\\
\>[3]{}\hsindent{2}{}\<[5]%
\>[5]{}\AK{where}{}\<[E]%
\\
\>[5]{}\hsindent{2}{}\<[7]%
\>[7]{}\Varid{aux}\;\mathbin{:}\;\forall\;(\AB{e}\;\mathbin{:}\;\AD{Expr})\;\to \;\forall\;(\Varid{x}\;\mathbin{:}\;\AD{Zipper\ensuremath{\Downarrow}}\;\AB{e}){}\<[E]%
\\
\>[7]{}\hsindent{4}{}\<[11]%
\>[11]{}\to \;\forall\;(\Varid{y}\;\mathbin{:}\;\AD{Zipper\ensuremath{\Downarrow}}\;\AB{e})\;\to \;\Varid{y}\;\AD{<}\;\Varid{x}\;\to \;\AD{Acc}\;(\Conid{IxltOp}\;\AB{e})\;\Varid{y}{}\<[E]%
\\
\>[5]{}\hsindent{2}{}\<[7]%
\>[7]{}\Varid{aux}\;\AS{.}\hspace*{-0.1cm}\;(\AI{Add}\;(\Varid{plug}))\;\AS{.}\hspace*{-0.1cm}\;((\AB{\ensuremath{t_2}}\;\Varid{,}\;\AI{Left}\;\AB{\ensuremath{t_1}'}\;\hspace{.4em}\AI{:\hspace{-.1em}:}\hspace{.4em}\;\AB{\ensuremath{s_2}}))\;\Varid{,}\;\AI{refl})\;\AS{.}\hspace*{-0.1cm}\;((\AB{\ensuremath{t_1}}\;\Varid{,}\;\AI{Right}\;\AB{n}\;\AB{\ensuremath{t_2}'}\;\AB{\ensuremath{eq}}\;\hspace{.4em}\AI{:\hspace{-.1em}:}\hspace{.4em}\;\AB{\ensuremath{s_1}})\;\Varid{,}\;\Varid{eq2})\;{}\<[E]%
\\
\>[7]{}\hsindent{4}{}\<[11]%
\>[11]{}(\AI{<\text{-}Right\text{-}Left}\;\Varid{eq1}\;\Varid{eq2})\;\mathrel{=}\;\{\mskip1.5mu \mathbin{!!}\mskip1.5mu\}\;{}\<[E]%
\\
\>[5]{}\hsindent{2}{}\<[7]%
\>[7]{}\Varid{aux}\;\Varid{...}{}\<[E]%
\ColumnHook
\end{hscode}\resethooks

We have now the proof of well foundedness for the relation defined
over top-down Zippers. We also have proven that there is an
equivalence between top-down and bottom-up Zippers. We exploit it by
using the top-down encoding for the termination proof while we use the
bottom-up to actually compute in a tail recursive manner.

Thus we prove a lemma stating that if we apply unload to a bottom-up
Zipper and this results in another Zipper, then the result is smaller
by the relation than the input. However, to show it we have to convert
them to the top down representation. In overall, what we have is the
following lemma:

\begin{hscode}\SaveRestoreHook
\column{B}{@{}>{\hspre}l<{\hspost}@{}}%
\column{13}{@{}>{\hspre}l<{\hspost}@{}}%
\column{E}{@{}>{\hspre}l<{\hspost}@{}}%
\>[B]{}\Varid{unload-ltop}\;\mathbin{:}\;\forall\;\AB{n}\;\AB{\ensuremath{eq}}\;\Varid{s}\;\Varid{t'}\;\Varid{s'}\;\to \;\AF{unload}\;(\Conid{Tip}\;\AB{n})\;(\Conid{TipA}\;\AB{n})\;\AB{\ensuremath{eq}}\;\Varid{s}\;\AD{\ensuremath{\equiv}}\;\Varid{inj₁}\;(\Varid{t'}\;\Varid{,}\;\Varid{s'}){}\<[E]%
\\
\>[B]{}\hsindent{13}{}\<[13]%
\>[13]{}\to \;(\Varid{t'}\;\Varid{,}\;\AF{reverse}\;\Varid{s'})\;\AD{\_<\_}\;(\AB{n}\;\Varid{,}\;\AF{reverse}\;\Varid{s}){}\<[E]%
\ColumnHook
\end{hscode}\resethooks

\subsection{Correctness}
\label{sec:basic-correctness}
Indexing the \emph{Zipper} with an expression allow us to prove
correcness of the transformation easily. The expression during the
fold does not change, thus in every step of the computation the result
of its evaluation remains constant.

By using induction over the definition of unload, we can prove that
when \ensuremath{\AF{unload}} delivers a value, it corresponds to the result of
evaluating of the input expression.  In order to do so, we enrich the
type of \ensuremath{\AF{unload}} to include the expression that has already been
folded and we have its result.

\begin{hscode}\SaveRestoreHook
\column{B}{@{}>{\hspre}l<{\hspost}@{}}%
\column{3}{@{}>{\hspre}l<{\hspost}@{}}%
\column{19}{@{}>{\hspre}l<{\hspost}@{}}%
\column{E}{@{}>{\hspre}l<{\hspost}@{}}%
\>[3]{}\AF{unload}\;\mathbin{:}\;(\AB{e}\;\mathbin{:}\;\AD{Expr})\;\to \;(\AB{n}\;\mathbin{:}\;\AD{\ensuremath{\mathbb{N}}})\;\to \;\AF{eval}\;\AB{e}\;\AD{\ensuremath{\equiv}}\;\AB{n}\;\to \;\AD{Stack}\;\to \;(\AD{\ensuremath{\mathbb{N}}}\;\AF{\ensuremath{\times}}\;\AD{Stack})\;\AD{\ensuremath{\uplus}}\;\AD{\ensuremath{\mathbb{N}}}{}\<[E]%
\\[\blanklineskip]%
\>[3]{}\Varid{unload-correct}\;{}\<[19]%
\>[19]{}\mathbin{:}\;\forall\;(\AB{e}\;\mathbin{:}\;\AD{Expr})\;(\AB{n}\;\mathbin{:}\;\AD{\ensuremath{\mathbb{N}}})\;(\AB{\ensuremath{eq}}\;\mathbin{:}\;\AF{eval}\;\AB{e}\;\AD{\ensuremath{\equiv}}\;\AB{n})\;(\Varid{s}\;\mathbin{:}\;\AD{Stack})\;(\Varid{x}\;\mathbin{:}\;\AD{\ensuremath{\mathbb{N}}}){}\<[E]%
\\
\>[19]{}\to \;\AF{unload}\;\AB{e}\;\AB{n}\;\AB{\ensuremath{eq}}\;\Varid{s}\;\Varid{≡}\;\AI{inj\ensuremath{_2}}\;\Varid{x}\;\to \;\AF{eval}\;\AB{e}\;\AD{\ensuremath{\equiv}}\;\Varid{x}{}\<[E]%
\ColumnHook
\end{hscode}\resethooks

Proving correctness of the whole transformation amounts to show that
it holds for the auxiliary recursor that we use to write the function
\ensuremath{\AF{tail\text{-}rec\text{-}eval}}. We use well founded recursion to do structural
recursion over the accesibility predicate and use the lemma
\ensuremath{\Varid{unload-correct}} in the base case.

\todo[inline]{STOP HERE}

\section{Regular universe}
\label{sec:universe}
  + Universe interpretation generic programming
  + Fixpoint
  + Example??
\section{Dissection}
\label{sec:dissection}  
  + Dissection in agda
  + Plug
  + Zipper up Zipper down
  + Make clear the separation between recursion in the functor level and
  the fix level
  + relation on dissection?

\section{Termination and correctness, generically}
\label{sec:dissection}  

  
\section{Conclusion and future work}

%% Acknowledgments
\begin{acks}                            %% acks environment is optional
                                        %% contents suppressed with 'anonymous'
  %% Commands \grantsponsor{<sponsorID>}{<name>}{<url>} and
  %% \grantnum[<url>]{<sponsorID>}{<number>} should be used to
  %% acknowledge financial support and will be used by metadata
  %% extraction tools.
  This material is based upon work supported by the
  \grantsponsor{GS100000001}{National Science
    Foundation}{http://dx.doi.org/10.13039/100000001} under Grant
  No.~\grantnum{GS100000001}{nnnnnnn} and Grant
  No.~\grantnum{GS100000001}{mmmmmmm}.  Any opinions, findings, and
  conclusions or recommendations expressed in this material are those
  of the author and do not necessarily reflect the views of the
  National Science Foundation.
\end{acks}


%% Bibliography
\bibliography{main}



%% Appendix
\appendix
\section{Appendix}

Text of appendix \ldots

\end{document}

%%% Local Variables:
%%% mode: latex
%%% TeX-master: t
%%% TeX-command-default: "lagda2pdf"
%%% End: 


